% Created 2010-01-31 Sun 12:52
\documentclass[11pt]{beamer}

      \mode<presentation>

      \usetheme{Antibes}

      \usecolortheme{lily}

      \beamertemplateballitem

      \setbeameroption{show notes}
      \usepackage[utf8]{inputenc}

      \usepackage[T1]{fontenc}

      \usepackage{hyperref}

      \usepackage{color}
      \usepackage{listings}
      \lstset{numbers=none,language=[ISO]C++,tabsize=4,
  frame=single,
  basicstyle=\small,
  showspaces=false,showstringspaces=false,
  showtabs=false,
  keywordstyle=\color{blue}\bfseries,
  commentstyle=\color{red},
  }

      \usepackage{verbatim}

      \institute{Hack Århus}
          
       \subject{CNC}



\title{A CNC machine in every home}
\author{Jacob Toft Pedersen}
\date{31 January 2010}

\begin{document}

\maketitle

\setcounter{tocdepth}{3}
\tableofcontents
\vspace*{1cm}


\section{Introduction}
\label{sec-1}




\section{Why}
\label{sec-2}

\subsection*{Inspiration}\begin{frame}[fragile]
\label{sec-2.1}

\includegraphics[width=0.97\textwidth,keepaspectratio]{pc-va.jpg}
\begin{itemize}
\item Heard about reprap and 3D fabrication
\item CCC talk ``Objects as software''
\end{itemize}
\end{frame}
\subsection*{A necessity}\begin{frame}[fragile]
\label{sec-2.2}

\includegraphics[width=0.7\textwidth,keepaspectratio]{plugs.jpg}
\begin{itemize}
\item The possibility to make life-critical necessities like bathroom
     plug and coffee-scoop.
\end{itemize}
\end{frame}
\subsection*{Soon every home will have one}\begin{frame}[fragile]
\label{sec-2.3}

\includegraphics[width=0.5\textwidth,keepaspectratio]{laser.jpg}
\begin{itemize}
\item The price of a laser printer in `85 is equivalent to the price of a
     professional fabber today!
\end{itemize}
\end{frame}
\subsection*{Can Ikea survive the revolution?}\begin{frame}[fragile]
\label{sec-2.4}

\begin{itemize}
\item When we were able to digitize and share music (MP3's) the music
     market changed
\item What will happen when we can digitize real life objects, share
     and replicate them ???
\end{itemize}
\end{frame}
\section{How}
\label{sec-3}


\subsection*{JGRO}\begin{frame}[fragile]
\label{sec-3.1}

\includegraphics[width=0.7\textwidth,keepaspectratio]{jgro.png}
\begin{itemize}
\item This is a CNC-router

\begin{itemize}
\item but since its a Cartesian robot its possible to make it into a
      rep-strap.
\end{itemize}

\item Easy to build (often there is a chicken / egg problem)
\item Threaded rods, skateboard bearings and MDF
\end{itemize}
\end{frame}
\subsection*{Huset}\begin{frame}[fragile]
\label{sec-3.2}

\includegraphics[width=0.7\textwidth,keepaspectratio]{huset.jpg}
\begin{itemize}
\item They had everything i needed (minus skateboard bearings)
\item They had expertise and were very helpful
\end{itemize}
\end{frame}
\subsection*{z-axis}\begin{frame}[fragile]
\label{sec-3.3}

\includegraphics[width=0.9\textwidth,keepaspectratio]{z-axis.jpg}

\end{frame}
\subsection*{Machine}\begin{frame}[fragile]
\label{sec-3.4}

\includegraphics[width=0.9\textwidth,keepaspectratio]{athome.jpg}

\end{frame}
\subsection*{Control}\begin{frame}[fragile]
\label{sec-3.5}

\includegraphics[width=0.7\textwidth,keepaspectratio]{kit.jpg}
\begin{itemize}
\item Based on steppers, bought a kit from China.
\item Hooked up to linuxCNC via lpt.

\begin{itemize}
\item I did fear the setup procedure - but it was a breeze
\end{itemize}

\end{itemize}
\end{frame}
\subsection*{GCode}\begin{frame}[fragile]
\label{sec-3.6}

\begin{lstlisting}[language=c]
G21 (Unit in mm)
G90 (Absolute distance mode)
G64 P0.01 (Exact Path 0.001 tol.)
G40 (Cancel diameter comp.)
G49 (Cancel length comp.)
T1M6 (Tool change to T1)
G0 
Z 0.100000 F 50.000000
X 0.000000 Y 0.000000 
G1 F100.000000 
Z 0.000000 F 50.000000
G1 F50.000000 
\end{lstlisting}
\begin{itemize}
\item LinuxCNC, understands / interprets g-codes.
\item g-codes are ``assembly'' for machines.
\item You can write it, but you would rather ``compile'' to it.
\end{itemize}
\end{frame}
\section{Then what?}
\label{sec-4}

\includegraphics[width=0.9\textwidth,keepaspectratio]{scoop.png}
\begin{itemize}
\item Log in to thingiverse.com

\begin{itemize}
\item find something useful
\item hit the print button.
\end{itemize}

\item I did suspect that I needed some time to learn the ins and outs of CAE/CAM/CAD
\item ``Boy was I in for  a surprise''
\end{itemize}
\subsection*{The software}\begin{frame}[fragile]
\label{sec-4.1}

\includegraphics[width=0.9\textwidth,keepaspectratio]{artcam.png}
\begin{itemize}
\item Open source programs, several projects. Requires some experience
     with CAE/CAM/CAD.

\begin{itemize}
\item dxf2gcode
\item HeeksCAD/HeeksCNC
\end{itemize}

\end{itemize}
\end{frame}
\section{Example "A logo sign"}
\label{sec-5}

\includegraphics[width=0.9\textwidth,keepaspectratio]{hack_first.jpg}
\begin{itemize}
\item Needed a new Logo
\item Idea

\begin{itemize}
\item Angled planes, to reflect different colors of light.
\end{itemize}

\end{itemize}
\subsection*{Design}\begin{frame}[fragile]
\label{sec-5.1}

\includegraphics[width=0.9\textwidth,keepaspectratio]{design.jpg}
\begin{itemize}
\item I only have 3 axis. Thus cutting an angled plane, will be
      influenced by tool-size.
\item Should it be holes, protruding cylinders, halfway through?
\item What should the angle be, where to put lights? \\
\end{itemize}
\end{frame}
\subsection*{Generating gcode}\begin{frame}[fragile]
\label{sec-5.2}

\includegraphics[width=0.9\textwidth,keepaspectratio]{inkscape.png}
\begin{itemize}
\item 3d CAD model and use software to convert it to gcode
\item Use a Vector-graphics based approach and a program to calculate the gcode?
\item The easy solution : height map.
\end{itemize}
\end{frame}
\subsection*{Running the gcode}\begin{frame}[fragile]
\label{sec-5.3}

\includegraphics[width=0.9\textwidth,keepaspectratio]{spiral.png}
\begin{itemize}
\item 3 tools : 12 mm two-flute, 6 mm two-flute 3mm spiral cutter
\item Tool shapes, Feed rates come back to hunt you.
\item Two flute 12 mm, not made for drilling,  3 mm spiral-cutter does not have a 3 mm neck
\item 1.5 mm apart, does still leave tool-marks, height mapped
     cylinders.
\end{itemize}
\end{frame}
\subsection*{Finishing up}\begin{frame}[fragile]
\label{sec-5.4}

\includegraphics[width=0.7\textwidth,keepaspectratio]{logo_raw.jpg}
\begin{itemize}
\item Then sanding, painting - still needs a light setup.
\end{itemize}
\end{frame}
\section{"I made this"}
\label{sec-6}

\includegraphics[width=0.65\textwidth,keepaspectratio]{first.jpg}
\begin{itemize}
\item Setup for the ``Virgin cutting'' / ``Jomfru fræsning''
\end{itemize}
\subsection*{First object}\begin{frame}[fragile]
\label{sec-6.1}

\includegraphics[width=0.8\textwidth,keepaspectratio]{first_obj.jpg}
\begin{itemize}
\item The very first 3d object
\end{itemize}
\end{frame}
\subsection*{Gears and signs}\begin{frame}[fragile]
\label{sec-6.2}

\includegraphics[width=0.8\textwidth,keepaspectratio]{assorted.jpg}

\end{frame}
\subsection*{X-mas present prototype}\begin{frame}[fragile]
\label{sec-6.3}

\includegraphics[width=0.8\textwidth,keepaspectratio]{mads1.jpg}

\end{frame}
\subsection*{X-mas present done}\begin{frame}[fragile]
\label{sec-6.4}

\includegraphics[width=0.8\textwidth,keepaspectratio]{mads2.jpg}

\end{frame}
\subsection*{X-mas present in action}\begin{frame}[fragile]
\label{sec-6.5}

\includegraphics[width=0.8\textwidth,keepaspectratio]{mads3.jpg}

\end{frame}
\subsection*{Magazine holder}\begin{frame}[fragile]
\label{sec-6.6}

\includegraphics[width=0.8\textwidth,keepaspectratio]{magazine.jpg}

\end{frame}
\subsection*{Raster Gandhi}\begin{frame}[fragile]
\label{sec-6.7}

\includegraphics[width=0.5\textwidth,keepaspectratio]{gandhi.jpg}

\end{frame}
\subsection*{Raster Jimi}\begin{frame}[fragile]
\label{sec-6.8}

\includegraphics[width=0.8\textwidth,keepaspectratio]{jimi.jpg}

\end{frame}
\subsection*{Sign Mads}\begin{frame}[fragile]
\label{sec-6.9}

\includegraphics[width=0.9\textwidth,keepaspectratio]{mads.jpg}

\end{frame}
\subsection*{Djørup}\begin{frame}[fragile]
\label{sec-6.10}

\includegraphics[width=0.8\textwidth,keepaspectratio]{djorup.jpg}

\end{frame}
\subsection*{Thomas painting}\begin{frame}[fragile]
\label{sec-6.11}

\includegraphics[width=0.8\textwidth,keepaspectratio]{thomas1.jpg}


\end{frame}
\section{Resources}
\label{sec-7}

\begin{itemize}
\item \href{http://media.ccc.de/browse/congress/2008/25c3-2781-en-objects_as_software_the_coming_revolution.html}{Objects as Software: The Coming Revolution (CCC talk)}
\item \href{http://www.linuxcnc.org/}{Linux CNC}

\begin{itemize}
\item \href{http://wiki.linuxcnc.org/cgi-bin/emcinfo.pl%3FCam}{List of CAD/CAM/CAE software}
\end{itemize}

\item \href{http://www.cnczone.com}{CNC zone ``The place for CNC information''}
\item \href{http://reprap.org}{Rep Rap}
\item \href{http://www.lav-det-selv.dk/Forum/afv/topicsview/aff/36/}{Lav det selv - CNC forum}
\item \href{http://rotand.dk/blog}{Jacobs seldomly updated blog}
\end{itemize}

\end{document}